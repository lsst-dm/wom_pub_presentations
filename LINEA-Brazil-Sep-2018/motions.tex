
\setlength\myleftmargin{0.5cm}

\thispagestyle{empty}
\frame[plain]{
\frametitle{\color{yellow} Stellar motions}
\color{yellow}
\setbeamercovered{invisible}
\begin{tabular}{p{1.0\textwidth}}

Stellar motions can be predicted into the future and calculated for times in the past when all 6 phase-space coordinates (3 positions, 3 velocities) are known for each star
\\ \pause
The evolutionary history of the Galaxy is recorded mainly in the halo, where incoming galaxies got stripped by our Galaxy and incorporated
\\
\vspace{-0.2cm}
\begin{center}
\includegraphics[scale=0.55,trim=0 2.2cm 0 0] {merger}
\end{center}

\\ \pause
In such processes, stars got spread over the whole sky but their energy and (angular) momenta were conserved. Thus, it is possible to work out, even now, which stars belong to which merger and to reconstruct the accretion history of the halo
{\tiny (de Bruijne)}
\end{tabular}

}
