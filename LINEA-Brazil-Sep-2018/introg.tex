
\beamersetaveragebackground{black}

\setlength\myleftmargin{0cm}

\thispagestyle{empty}
\frame[plain]{
 \centering
 \vspace{-25pt}
    \includegraphics[scale=0.50]{images/tinto}

}

\thispagestyle{empty}
\frame[plain]{
\frametitle{\color{yellow} Milky way analogue}
\vspace{-3.3cm}\hspace{-10pt}\includegraphics[scale=0.4] {images/ngc1232}
}
\thispagestyle{empty}
\frame[plain]{
\frametitle{\color{yellow} Milky way analogue}
\vspace{-3.3cm}\hspace{-10pt}\includegraphics[scale=0.4] {images/ngc1232}
\vspace {-7cm}
{\color{yellow}
%\rowcolors{1}{blue!20}{blue!20}
\begin{tabular}{p{\textwidth}}
{\huge Milky way analogue}\\
\\
\\
\\
\\
\\
 \hspace{30pt}{\huge `Our sun'$\rightarrow$ \normalsize }\\

{ Our view is severely obstructed by the dust in the disk and relatively little is known about the origin, history, and structure of our own Galaxy}\\
\end{tabular}
\vspace{-4cm}
\begin{tabular} {r}
\\
 {We are unraveling the formation, composition, and evolution of the Galaxy}\\

\hspace{11cm} {\tiny Jos de Bruijne \normalsize }
\end{tabular}
}}





\frame[plain]{\frametitle{Origin of the Milky Way }
\vspace{-15pt}
\begin{center}
%only seems to work wth avi in same directory as pdf ..
   % \movie[width=9cm]{\includegraphics[width=9cm]{Aq-Amina-xz_100kpc_n-frame04sec}}{Aq.mov}
    \href {file:/Aq.mov} { \includegraphics[width=9cm]{Aq-Amina-xz_100kpc_n-frame04sec}}
\end{center}
\vspace{-15pt}
  {\scriptsize \href{run:Aq.mov}{Movie credit: Amina Helmi, University of Groningen}\hfill slides from Anthony Brown, Leiden University}

}
%\setbeamertemplate{frametitle}{{\centering \logoleft \hfill \headercenter \hfill \logoright}}

\frame{\frametitle{Origin of the Milky Way }
\vspace{-15pt}
\begin{center}
    \includegraphics[height=7cm]{Aq-Amina-xz_100kpc_n-frame50sec.png}
	\hspace{1cm}
    \includegraphics[height=7cm]{ngc5907_small.jpg}
  \end{center}
  \vskip-1.0cm
  {\color{yellow} \hspace{11cm}\scriptsize Image credit: R.\ Jay GaBany}
  \vspace{-1.0cm}
  {\scriptsize Just published using DR2 \cite{2041-8205-860-1-L11}}
  }



\thispagestyle{empty}
\beamersetaveragebackground{black}
\frame[plain]{ \frametitle{Beyond our galaxy}
	\vspace{-13pt}
\includegraphics[width=\textwidth,trim=0 0 0 0] {images/planckhistory}\\
	\vspace{-08pt}
\begin{columns}
\column{0.38\textwidth}
\includegraphics[width=1.0\textwidth,trim=0 0 0 0] {images/DarkMatterPie}\\
\column{0.62\textwidth}
\\
\vspace{-60pt}
{\color{yellow} The modern cosmological models can explain all observations, but need to {\em postulate} dark matter and dark energy (though gravity model could be wrong, too) }
\end{columns}
}





\frame{
\frametitle{Hipparchus to ESA's Hipparcos }
\begin{columns}
\column{0.20\textwidth}
\vspace{2cm}
\includegraphics[width=\textwidth,trim=0cm 0cm 0pt 0pt]{images/mp_starmap}

\column{0.8\textwidth}
\vspace{-0.5cm}
\includegraphics[width=\textwidth]{microas}\\
{
Gaia will take us to the next order of magnitude the microarcsecond.\\
e.g. A euro coin on the moon viewed from earth \\
\vspace{10pt}
Hipparcos measured $10^5$ objects : Gaia measures  $10^9$ \\
\vspace{10pt}
$\leftarrow$  superb account of the Hipparcos mission.
}
\end{columns}
}

\beamersetaveragebackground{white}





\begin{agaframe}{Science topics - surveys}
\begin{tikzpicture}
    \node (im) {\includegraphics[height=1.0\textheight]{gaiaim/Science-return.png}};
    \node  at (im) {\includegraphics[height=2.20cm, angle=-12]{gaiaim/Gaia-model-2012.png}};
    \node (pl) [right=-0.5cm of im] {\includegraphics[width=9cm] {images/planckhistory}};
  \end{tikzpicture}
\end{agaframe}




\frame{\frametitle{{\color{red} Killer asteroids}: the impact probability is not 0! }
\begin{columns}
\column{0.5\textwidth}
\includegraphics[width=1.0\textwidth,trim=0 0 0 0] {images/asteroidImpacts}\\

\column{0.5\textwidth}
\\
\vspace{-7cm}
{\bf LSST is the only survey capable of delivering completeness specified in the 2005 USA Congressional NEO mandate to NASA (to find 90\% NEOs larger than 140m)}\\
\vspace{1cm}
\includegraphics[width=1.0\textwidth,trim=0 0 0 0] {images/BarringerCrater}\\
\vspace{3mm}
The Barringer Crater, Arizona:    a 40m object 50,000 yr. ago

\end{columns}
}

\frame{\frametitle{ The Era Of Surveys }
\begin{columns}
\column{0.65\textwidth}
\includegraphics[width=1.0\textwidth,trim=0 0 0 0] {images/AstronomerCartoon}\\
\column{0.35\textwidth}
\\
\vspace{-7cm}
\includegraphics[width=1.0\textwidth,trim=0 0 0 0] {images/HSTpubs}\\
{\tiny \url{https://archive.stsci.edu/hst/bibliography/pubstat.html}}\\

{\small \ldots indicates  archival  research  probably play an important role in the scientific success of XMM-Newton
\cite{2014AN....335..210N}}

\end{columns}
}

