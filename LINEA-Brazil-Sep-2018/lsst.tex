\frame { \frametitle{ LSST:uniform sky survey }
\begin{columns}
\column{0.5\textwidth}
\vspace {-5cm}
 \\
An optical/near-IR survey of half the sky in ugrizy bands to r~27.5 (36 nJy) based on 825 visits over  a 10-year period: {\em deep wide fast}.
%It’s about 5,000 sq. deg. per night, *twice*, on
%average. That is, about 1,000 visits per night on average. You return to
%the same position on the sky in about 3-4 nights (in any band). Btw, it’s
%a nice coincidence worth remembering that the total exposure time per
%position over 10 years (in all bands) is equal to about one observing night.
\begin{itemize}
\item 90\% of time  spent on  uniform survey: every 3-4 nights, the whole observable sky scanned twice per night
\item	~100 PB of data: about a billion 16 Mpix images, enabling measurements\\ {\color{cyan} for 40 billion objects! }
\end{itemize}
{\tiny see also \url{http://www.lsst.org} and \cite{2008arXiv0805.2366I}-arXiv:0805.2366}

\column{0.5\textwidth}
	 \includegraphics[width=1.1\textwidth]{coverage.jpg}\\
{\bf 10-year simulation of LSST survey: number of visits in u,g,r band (Aitoff projection of eq. coordinates) }\\
\end{columns}

}
\frame { \frametitle{ LSST Camera }
\begin{columns}
\column{0.65\textwidth}
 \includegraphics[width=1.0\textwidth]{images/camera}
\column{0.35\textwidth}
 \\
\vspace {-5cm}
{\large \bf The largest astronomical camera:}
\begin {itemize}
\item 2800 kg
\item 3.2 Gpix
\end {itemize}
\end{columns}
}


\frame { \frametitle{ Science rafts }
\begin{columns}
\column{0.65\textwidth}
\vspace {-7cm}
\\
 \includegraphics[width=0.5\textwidth]{images/fov}
 \includegraphics[width=0.5\textwidth]{images/raftCCD}\\

Modular design: 3200 Megapix = 189 x16 Megapix CCD\\
9 CCDs share electronics: raft  (=camera 144 Megapix)\\
\hfill {\bf First of 21 rafts available $\longrightarrow$} \\
\column{0.25\textwidth}
	 \includegraphics[width=1.0\textwidth]{images/raftTower}
\end{columns}
}



\frame { \frametitle{ Hardware arriving }
\begin{columns}
\column{0.6\textwidth}
\vspace{-3cm}
\begin{itemize}
\item Fused silica optics
	\begin{itemize}
	\item contract -  Ball Aerospace (With AOS and Vanguard) , TSESO, REOSC and Materion
	\item L1 ready to polish - L2 being polished $\Longrightarrow$
	\item L3 coming this year
	\end{itemize}
\item Cryostat to keep cold CCDs at -100C
	\begin{itemize}
	\item Grid machining and cell mockup $\Longrightarrow$
	\item Awarded Housing \& support cylinder fabrication.
	\end{itemize}
\end{itemize}
\column{0.3\textwidth}
	 \includegraphics[width=1.0\textwidth]{images/l2Optic}\\
	 \includegraphics[width=1.0\textwidth]{images/grid}\\
\end{columns}

}

\frame { \frametitle{ Site shaping up (July)}
	\vspace{-2pt}
	\includegraphics[width=1.0\textwidth,trim=0cm 4cm 0 2cm,clip]{images/cerroJul2018}

\begin{itemize}
\item
\end{itemize}
}

\frame {\frametitle{ Marine tracking and logisticsi (Sept 1)}
\begin{columns}
\column{0.65\textwidth}
      \includegraphics[width=1.0\textwidth]{images/MarineTracking}\\
\column{0.35\textwidth}
 \\
\vspace{-6cm}

\begin{itemize}
\item Mv Valparaiso Express – 5 containers held up in Peru, transshipment required to MN Callao Express
MN Callao Express – 10 Containers
\item BBC Arizona – Late to Antwerp for Coating vessel and 7 crates
\item K\&N warehouse Ready in Chile for 15 Containers
\item Los Angeles Warehouse ready for M2
\item 9 Pallets of M1M3 lift fixture hardware is MIA! Believed to be in Phoenix warehouse – Chapter 7 Trustee Custody
\item 6 EIE Containers in Italy scheduled for Packing in Sept.
\end{itemize}

\end{columns}
}

\frame {\frametitle{  DM build and deploy }
	\vspace{-1cm}
	\begin{columns}
		\column{0.33\textwidth}
		\begin{center}
			      \includegraphics[width=1.0\textwidth]{images/DMSDeployment}\\
		      \end{center}
		      \column{0.6\textwidth}
		       \\
		       \vspace{1cm}
		       DM must build everything to get LSST products (see \url{http://ls.st/dpdd})  to the users.
		       \begin{itemize}
			       \item large data sets (20TB/night)
			       \item complex analysis
			       \item aiming for small systematics
			       \item Science Alerts in under 2 minutes .. (aiming for 1 minute)
		       \end{itemize}
		       About $1\over{2}$  million lines of code (C++/python) all open source on github\\
		       \vspace{25pt}
		       {\tiny \bf diagram K.T. Lim}
	       \end{columns}
       }



\frame {\frametitle{It all sits on machines  }
\begin{columns}
\column{0.4\textwidth}
\begin{center}
      \includegraphics[width=1.0\textwidth]{images/pdac}\\
      Prototype Data Access Center Machines at NCSA
\end{center}
\column{0.6\textwidth}
\\
\vspace{-2mm}
\begin{itemize}
   \item  NCSA
\begin{itemize}
	\item  GPFS  2 PB  (Lots more coming soon!)
    \item Common batch Computing - 2304 cores ($48\times 48$)
    \item use of common NCSA VSphere infrastructure
    \item NCSA tape commons (currently in Blue Waters)
	\item Fast (100Gbs) links to ESNET,I2,MREN.
\end{itemize}
\item Supporting :
\begin{itemize}
	\item   Developer spaces and experimentation (Kubernetes), PDAC, etc.
    \item  LSST Level one test stand (OCS simulator, WAN Emulator, EFD prototype).
\end{itemize}
\item Currently  Amazon for builds .
\item IN2P3 - full QSERV
\end{itemize}

\end{columns}
}




