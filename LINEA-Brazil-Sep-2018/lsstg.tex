\frame { \frametitle{ LSST:uniform sky survey }
\begin{columns}
\column{0.5\textwidth}
\vspace {-6.5cm}
 \\
An optical/near-IR survey of half the sky in ugrizy bands to r~27.5 (36 nJy) based on 825 visits over  a 10-year period: {\em deep wide fast}.
%It’s about 5,000 sq. deg. per night, *twice*, on
%average. That is, about 1,000 visits per night on average. You return to
%the same position on the sky in about 3-4 nights (in any band). Btw, it’s
%a nice coincidence worth remembering that the total exposure time per
%position over 10 years (in all bands) is equal to about one observing night.
\begin{itemize}
\item 90\% of time  spent on  uniform survey: every 3-4 nights, the whole observable sky scanned twice per night
\item	~100 PB of data: about a billion 16 Mpix images, enabling measurements\\ {\color{cyan} for 40 billion objects! }
\end{itemize}
{\tiny see also \url{http://www.lsst.org} and \cite{2008arXiv0805.2366I}-arXiv:0805.2366}\\
Call for white papers - \url {https://www.lsst.org/call-whitepaper-2018}

\column{0.5\textwidth}
	 \includegraphics[width=1.1\textwidth]{coverage.jpg}\\
{\tiny \bf 10-year simulation of LSST survey: number of visits in u,g,r band (Aitoff projection of eq. coordinates) }\\
\end{columns}

}
\frame { \frametitle{ LSST Camera }
\begin{columns}
\column{0.65\textwidth}
 \includegraphics[width=1.0\textwidth]{images/camera}
\column{0.35\textwidth}
 \\
\vspace {-5cm}
{\large \bf The largest astronomical camera:}
\begin {itemize}
\item 2800 kg
\item 3.2 Gpix
\end {itemize}
\end{columns}
}


\frame { \frametitle{ Site shaping up (July 2018)}
\begin{center}
	\vspace{-10pt}
	\includegraphics[width=0.9\textwidth,trim=0cm 0cm 0 0cm,clip]{images/cerroJul2018}\\
\end{center}
\vspace{-8cm}
\hspace{1cm}{\tiny \color{yellow}  \url{http://ls.st/8p0}}\\
}

\frame{\frametitle{LSST Project Schedule  }
\begin{center}
\includegraphics[width=0.9\textwidth]{images/ProjectSchedule}
\end{center}
}


