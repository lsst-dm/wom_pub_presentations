
\beamersetaveragebackground{black}

\thispagestyle{empty}
\frame[plain]{
\frametitle{\color{yellow} Milky way analogue}
\setbeamercovered{invisible}
\vspace{-3.3cm}\hspace{-10pt}\includegraphics[scale=0.4] {images/ngc1232}
\vspace {-9cm}
{\color{yellow}
%\rowcolors{1}{blue!20}{blue!20}
\begin{tabular}{p{\textwidth}}
\onslide<1->{\huge Milky way analogue {\tiny NGC1232}}\\
\\
\\
\\
\\
\\
\onslide<1-> \hspace{30pt}{\huge `Our sun'$\rightarrow$ \normalsize }\\

\onslide<2->{ Our view is severely obstructed by the dust in the disk and relatively little is known about the origin, history, and structure of our own Galaxy}\\
\end{tabular}
\vspace{-4cm}
\begin{tabular} {r}
\\
\onslide<2-> {Gaia's main aim: unravel the formation, composition, and evolution of the Galaxy}\\

\hspace{11cm} {\tiny Jos de Bruijne \normalsize }
\end{tabular}
}}


\frame{\frametitle{Some numbers and scales  }
\begin{itemize}
\item   {\bf K}ilo = 1000  = $10^{3}$  {\bf M}ega = 1000000 = $10^{6}$ {\bf G}iga = 1000000000 = $10^{9}$ {\bf T}era = $10^{12}$
\item But note in computing  ..  KB = 1024 Bytes .. $2^{10}$
\pause
\item  $ 1 \deg$ (degree) = $60 \arcmin$ (arc minutes), $1 \arcmin = 60 \arcsec$ (arc second)
\pause
\item 1 AU (astronomical unit) distance of earth from sun $\sim  1.496 10^{11} $ Meters
\item 1 light year $\sim 9.461 \times 10^{15}$ Meters  $\sin $63241 AU
\pause
\item Parallax of $1\arcsec$ known as 1 parsec  $\sim$  3.26 Light years  $\sim$ 206165AU
\item $3.26 \text{light years} = 3.26 \times 9.461 \times 10^{15} \sim 3.084 \times 10^{16} $Meters
\end{itemize}
}



\frame{\frametitle{Another Milky way Analogue (NGC4565)}
\vspace{-1.1cm}
\centering
\includegraphics[width=0.55\textwidth,angle=-90]{ngc4565}

%NGC 4565, ngc4565_mclaughlin_big.jpg
%Needle Galaxy in Coma, William McLaughlin, 50 million light years
}





\frame[plain]{\frametitle{Origin of the Milky Way }
\vspace{-15pt}
\begin{center}
%only seems to work wth avi in same directory as pdf ..
   % \movie[width=9cm]{\includegraphics[width=9cm]{Aq-Amina-xz_100kpc_n-frame04sec}}{Aq.mov}
    \href {file:/Aq.mov} { \includegraphics[width=9cm]{Aq-Amina-xz_100kpc_n-frame04sec}}
\end{center}
\vspace{-15pt}
  {\scriptsize \href{run:Aq.mov}{Movie credit: Amina Helmi, University of Groningen}\hfill slides from Anthony Brown, Leiden University}

}
%\setbeamertemplate{frametitle}{{\centering \logoleft \hfill \headercenter \hfill \logoright}}

\frame{\frametitle{Origin of the Milky Way }
\vspace{-15pt}
\begin{center}
    \includegraphics[height=7cm]{Aq-Amina-xz_100kpc_n-frame50sec.png}
	\hspace{1cm}
    \includegraphics[height=7cm]{ngc5907_small.jpg}
  \end{center}
  \vskip-1.0cm
  {\color{yellow} \hspace{11cm}\scriptsize Image credit: R.\ Jay GaBany}
  \vspace{-1.0cm}
  {\scriptsize Just published using DR2 \cite{2041-8205-860-1-L11}}
  }



\frame[plain]{ \frametitle{Beyond our galaxy}
	\vspace{-33pt}
\includegraphics[width=\textwidth,trim=0 0 0 0] {images/planckhistory}\\
	\vspace{-08pt}
\begin{columns}
\column{0.38\textwidth}
\includegraphics[width=1.0\textwidth,trim=0 0 0 0] {images/DarkMatterPie}\\
\column{0.62\textwidth}
\\
\vspace{-60pt}
{ The modern cosmological models can explain all observations, but need to {\em postulate} dark matter and dark energy (though gravity model could be wrong, too) }
\end{columns}
}





\frame{
\frametitle{Hipparchus to ESA's Hipparcos }
\begin{columns}
\column{0.20\textwidth}
\vspace{2cm}
\includegraphics[width=\textwidth,trim=0cm 0cm 0pt 0pt]{images/mp_starmap}

\column{0.8\textwidth}
\vspace{-0.5cm}
\includegraphics[width=\textwidth]{microas}\\
{
Gaia will take us to the next order of magnitude the microarcsecond.\\
e.g. A euro coin on the moon viewed from earth \\
\vspace{10pt}
Hipparcos measured $10^5$ objects : Gaia measures  $10^9$ \\
\vspace{10pt}
$\leftarrow$  superb account of the Hipparcos mission.
}
\end{columns}
}

\beamersetaveragebackground{white}




\frame{\frametitle{Some numbers and scales  }
\begin{itemize}
\item   {\bf K}ilo = 1000  = $10^{3}$  {\bf M}ega = 1000000 = $10^{6}$ {\bf G}iga = 1000000000 = $10^{9}$ {\bf T}era = $10^{12}$
\item But note in computing  ..  KB = 1024 Bytes .. $2^{10}$
\pause
\item  $ 1 \deg$ (degree) = $60 \arcmin$ (arc minutes), $1 \arcmin = 60 \arcsec$ (arc second)
\pause
\item 1 AU (astronomical unit) distance of earth from sun $\sim  1.496 10^{11} $ Meters
\item 1 light year $\sim 9.461 \times 10^{15}$ Meters  $\sin $63241 AU
\pause
\item Parallax of $1\arcsec$ known as 1 parsec  $\sim$  3.26 Light years  $\sim$ 206165AU
\item $3.26 \text{light years} = 3.26 \times 9.461 \times 10^{15} \sim 3.084 \times 10^{16} $Meters
\end{itemize}
}



\begin{agaframe}{Science topics - surveys}
\begin{tikzpicture}
    \node (im) {\includegraphics[height=1.0\textheight]{gaiaim/Science-return.png}};
    \node  at (im) {\includegraphics[height=2.20cm, angle=-12]{gaiaim/Gaia-model-2012.png}};
    \node (pl) [right=-0.5cm of im] {\includegraphics[width=9cm] {images/planckhistory}};
  \end{tikzpicture}
\end{agaframe}




\frame{\frametitle{{\color{red} Killer asteroids}: the impact probability is not 0! }
\begin{columns}
\column{0.5\textwidth}
\includegraphics[width=1.0\textwidth,trim=0 0 0 0] {images/asteroidImpacts}\\

\column{0.5\textwidth}
\\
\vspace{-7cm}
{\bf LSST is the only survey capable of delivering completeness specified in the 2005 USA Congressional NEO mandate to NASA (to find 90\% NEOs larger than 140m)}\\
\vspace{0.5cm}
\includegraphics[width=1.0\textwidth,trim=0 0 0 0] {images/BarringerCrater}\\
\vspace{3mm}
The Barringer Crater, Arizona:    a 40m object 50,000 yr. ago

\end{columns}
}
