\section{Astrometry}
\frame {\frametitle{ Astrometric Solution}
Just one part of the Gaia processing !\\
From the Hipparcos catalogue
\cite[Volume 3 Chapter 23]{hip:catalogue}.

\begin{block}{Minimisation problem for astrometry}
\begin{equation}
	 ^{min}_{\bfvec{a},\bfvec{n}} \| \bfvec{g^{obs}} - \bfvec{g^{calc}}(\bfvec{a},\bfvec{n})\|_{M}
	\label{eq:generalobs}
\end{equation}
\end{block}
\begin{itemize}%[<+->]
\item $\bfvec{a}$  is the vector of unknowns describing a star's barycentric motion
represented by the measurement vector $\bfvec{g_{k}} = (G_{k},H_{k})^{\prime}$
and associated statistics.
\item $\bfvec{g^{obs}}$ represents the vector of all measurements
\item  $\bfvec{g^{calc}}$ represents the vector of detector
coordinates calculated from the astrometric parameters.
\item $\bfvec{n}$ is a vector
of nuisance parameters - required for realistic modeling (e.g. attitude, instrument calibration)

\item   $M$ metric defined by the statistics
of the data, (error weighting)

\end{itemize}
The complete new formulation for Gaia is in \citep{2012A&A...538A..78L}.
}


\frame {\frametitle{ Basic Gaia Problem }
%\vspace{-10pt}
Put more simply the data reduction must:\\
{\em
find the astrometric parameters (catalogue) best
predicting the ($10^{12}$) focal plane observations of the sources.
\citep{2011ExA....31..215O}
}
\begin{center}
\includegraphics[scale=0.6, clip=true,trim=1cm 0cm 0 1.5cm ]{gaiaim/predict}
\end{center}
}

\frame {\frametitle{ Look at one block: Source modeling }
%\vspace{-15pt}
The mapping or modeling  of
the observables
 $\bfvec{g}$ is done by three successive transformations:
 \begin{enumerate}%[<+->]
\item from astrometric parameters to the celestial directions of a star at the
instant of observation, using an astrometric model {\color{blue}\bf S}

\item from celestial  to instrument frame directions using an attitude
model {\color{blue}\bf A}
\item and finally from instrument directions to detector coordinates using an
instrument model {\color{blue}\bf C}
 \end{enumerate}
\vspace{-25pt}
\begin{center}
\includegraphics[scale=0.38,clip=true,trim=1cm 0cm 12cm 0 ]{gaiaim/atttoon}
\includegraphics[scale=0.38,clip=true,trim=3cm 0cm 6cm 0 ]{gaiaim/pixcoord}
\end{center}
}



\frame {\frametitle{ Source Update }
\setbeamercovered{invisible}
%\vspace{-14pt}
We fit the model to the observations:
\begin{block}{ Least squares for source update}
\begin{equation}\label{eq:lsobsmatrix}
\bfvec{Ax} \sim \bfvec{b} \pm \bfvec{\sigma}
\end{equation}
\begin{equation}
 \text{where~} \bfvec{b}_{i} = \bfvec{y}_{i}- f_{i}(\bfvec{a,q})
 \label{eq:f}
\end{equation}
\end{block}
Here $\bfvec{y}_{i}$ are the observed field angles  $f_{i}$ is a function to calculates field angles
from the current model.\\
\pause
In java (SourceUpdateCalculatorWrapper):
\includegraphics[width=\textwidth]{gaiaim/srcup}\\
\citep{2011ExA....31..215O}
}


\frame{\frametitle{Its all team work on big projects }
\centering
\includegraphics[width=\textwidth]{gaiaim/agis17}\\
}

