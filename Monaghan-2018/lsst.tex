\frame { \frametitle{ LSST:uniform sky survey }
\begin{columns}
\column{0.5\textwidth}
\vspace {-5cm}
 \\
An optical/near-IR survey of half the sky in ugrizy bands to r~27.5 (36 nJy) based on 825 visits over  a 10-year period: {\em deep wide fast}.
%It’s about 5,000 sq. deg. per night, *twice*, on
%average. That is, about 1,000 visits per night on average. You return to
%the same position on the sky in about 3-4 nights (in any band). Btw, it’s
%a nice coincidence worth remembering that the total exposure time per
%position over 10 years (in all bands) is equal to about one observing night.
\begin{itemize}
\item 90\% of time  spent on  uniform survey: every 3-4 nights, the whole observable sky scanned twice per night
\item	~100 PB of data: about a billion 16 Mpix images, enabling measurements\\ {\color{cyan} for 40 billion objects! }
\end{itemize}
{\tiny see also \url{http://www.lsst.org} and \cite{2008arXiv0805.2366I}-arXiv:0805.2366}

\column{0.5\textwidth}
	 \includegraphics[width=1.1\textwidth]{images/coverage}\\
{\bf 10-year simulation of LSST survey: number of visits in u,g,r band (Aitoff projection of eq. coordinates) }\\
\end{columns}

}
\frame { \frametitle{ LSST Camera }
\begin{columns}
\column{0.65\textwidth}
 \includegraphics[width=1.0\textwidth]{images/camera}
\column{0.35\textwidth}
 \\
\vspace {-5cm}
{\large \bf The largest astronomical camera:}
\begin {itemize}
\item 2800 kg
\item 3.2 Gpix
\end {itemize}
\end{columns}
}

\frame { \frametitle{ Site shaping up }
\centering
	\includegraphics[width=0.7\textwidth,trim=0cm 0cm 0 0cm,clip]{images/cerroMay2018}

}

\frame {\frametitle{ Data management }
\begin{columns}
\column{0.7\textwidth}
      \includegraphics[width=1.0\textwidth]{images/dm2018}\\
\column{0.3\textwidth}
 \\
\vspace{-5cm}

 DM Mission :\\
    {\em  Stand up operable, maintainable, quality services to deliver high-quality LSST data products for science, all on time and within reasonable cost.}\\
\end{columns}
    \vspace{7pt}
LSST DM development is distributed across the Americas.\\

{\color{blue} Plus we have partners like IN2P3}

}

\frame {\frametitle{SDSS image }
	\vspace{-1mm}
	\begin{columns}
		\column{0.5\textwidth}
		\vspace{-2mm}
	      \includegraphics[width=1.0\textwidth]{images/SDSScosmos}\\
		      \column{0.5\textwidth}
		      \\
		       \vspace{1cm}
	       Nice colours \cite{2004PASP..116..133L}\\
	       $\approx  3.5 \arcmin$\\
		       \vspace{20mm}
	       {\tiny \bf{Image  Robert Lupton}}
	       \end{columns}
}
\frame {\frametitle{Hyper Suprime Cam (HSC) on Subaru }
	\vspace{-1mm}
	\begin{columns}
		\column{0.5\textwidth}
		\vspace{-2mm}
	      \includegraphics[width=1.0\textwidth]{images/HSCcosmos}\\
		      \column{0.5\textwidth}
		       \\
		       \vspace{1cm}
		       HSC image (COSMOS) from g,r(1.5 hrs) ,i(3 hrs) PSF matched co-add ($\approx 27.5$)\\
		       \vspace{3mm}
		       Processed with  {\em LSST Stack} \url{https://pipelines.lsst.io/}\\
		       \vspace{10mm}
	       {\tiny \bf{Image HSC collaboration,  Robert Lupton}}
	       \end{columns}
       }




