\addtocounter{table}{-1}
\begin{longtable}{|l|p{0.8\textwidth}|}\hline
\textbf{Acronym} & \textbf{Description}  \\\hline

AGIS & Astrometric Global Iterative Solution \\\hline
Archive & The repository for documents required by the NSF to be kept. These include documents related to design and development, construction, integration, test, and operations of the LSST observatory system. The archive is maintained using the enterprise content management system DocuShare, which is accessible through a link on the project website www.project.lsst.org. \\\hline
CAOM & Common Astronomical Observation Model \\\hline
CCD & Charge-Coupled Device \\\hline
CI & Continuous Integration \\\hline
CU & Coordination Unit \\\hline
Camera & The LSST subsystem responsible for the 3.2-gigapixel LSST camera, which will take more than 800 panoramic images of the sky every night. SLAC leads a consortium of Department of Energy laboratories to design and build the camera sensors, optics, electronics, cryostat, filters and filter exchange mechanism, and camera control system. \\\hline
DM & Data Management \\\hline
DMTR & DM Test Report \\\hline
Data Backbone & The software that provides for data registration, retrieval, storage, transport, replication, and provenance capabilities that are compatible with the Data Butler. It allows data products to move between Facilities, Enclaves, and DACs by managing caches of files at each endpoint, including persistence to long-term archival storage (e.g. tape). \\\hline
Data Management & The LSST Subsystem responsible for the Data Management System (DMS), which will capture, store, catalog, and serve the LSST dataset to the scientific community and public. The DM team is responsible for the DMS architecture, applications, middleware, infrastructure, algorithms, and Observatory Network Design. DM is a distributed team working at LSST and partner institutions, with the DM Subsystem Manager located at LSST headquarters in Tucson. \\\hline
Data Release & The approximately annual reprocessing of all LSST data, and the installation of the resulting data products in the LSST Data Access Centers, which marks the start of the two-year proprietary period. \\\hline
ESA & European Space Agency \\\hline
ESAC & European Space Astronomy Centre \\\hline
ESOC & European Space Operations Centre \\\hline
ESTEC & European Space Technology Engineering Centre \\\hline
FGCM &  Forward Global Calibration Model \\\hline
GAVIP & Gaia Added Value Interface Platform \\\hline
GB & Gigabyte \\\hline
HEALPix & Hierarchical Equal-Area iso-Latitude Pixelisation \\\hline
HEASARC & NASA's Archive of Data on Energetic Phenomena \\\hline
HSC & Hyper Suprime-Cam \\\hline
HTM & Hierarchical Triangular Mesh \\\hline
IPAC & No longer an acronym; science and data center at Caltech \\\hline
IR & Infra Red \\\hline
ISSI & International Space Science Institute \\\hline
IVOA & International Virtual-Observatory Alliance \\\hline
LDM & LSST Data Management (Document Handle) \\\hline
LSE & LSST Systems Engineering (Document Handle) \\\hline
LSST & Large Synoptic Survey Telescope \\\hline
MPP & Massively Parallel Process \\\hline
NASA & National Aeronautics and Space Administration \\\hline
NCSA & National Center for Supercomputing Applications \\\hline
Operations & The 10-year period following construction and commissioning during which the LSST Observatory conducts its survey \\\hline
PB & PetaByte \\\hline
PSF & Point Spread Function \\\hline
Project Manager & The person responsible for exercising leadership and oversight over the entire LSST project; he or she controls schedule, budget, and all contingency funds \\\hline
Qserv & Proprietary Database built by SLAC for LSST \\\hline
SDSS & Sloan Digital Sky Survey \\\hline
Science Platform & A set of integrated web applications and services deployed at the LSST Data Access Centers (DACs) through which the scientific community will access, visualize, and perform next-to-the-data analysis of the LSST data products. \\\hline
Subsystem & A set of elements comprising a system within the larger LSST system that is responsible for a key technical deliverable of the project. \\\hline
TAP & Table Access Protocol \\\hline
TB & TeraByte \\\hline
XMM & X-ray Multi-mirror Mission (ESA; officially known as XMM-Newton) \\\hline
arcmin & arcminute minute of arc (unit of angle) \\\hline
arcsec & arcsecond second of arc (unit of angle) \\\hline
astrometry & In astronomy, the sub-discipline of astrometry concerns precision measurement of positions (at a reference epoch), and real and apparent motions of astrophysical objects. Real motion means 3-D motions of the object with respect to an inertial reference frame; apparent motions are an artifact of the motion of the Earth. Astrometry per se is sometimes confused with the act of determining a World Coordinate System (WCS), which is a functional characterization of the mapping from pixels in an image or spectrum to world coordinate such as (RA, Dec) or wavelength. \\\hline
calibration & The process of translating signals produced by a measuring instrument such as a telescope and camera into physical units such as flux, which are used for scientific analysis. Calibration removes most of the contributions to the signal from environmental and instrumental factors, such that only the astronomical component remains. \\\hline
camera & An imaging device mounted at a telescope focal plane, composed of optics, a shutter, a set of filters, and one or more sensors arranged in a focal plane array. \\\hline
metric & A measurable quantity which may be tracked. A metric has a name, description, unit, references, and tags (which are used for grouping). A metric is a scalar by definition. See also: aggregate metric, model metric, point metric. \\\hline
\end{longtable}
