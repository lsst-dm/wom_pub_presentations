\frame { \frametitle{ LSST:uniform sky survey }
\begin{columns}
\column{0.45\textwidth}
\vspace {-0.3cm}
 \\
An optical/near-IR survey of half the sky in ugrizy bands to r~27.5 (36 nJy) based on 825 visits over  a 10-year period: {\em deep wide fast}.
%It’s about 5,000 sq. deg. per night, *twice*, on
%average. That is, about 1,000 visits per night on average. You return to
%the same position on the sky in about 3-4 nights (in any band). Btw, it’s
%a nice coincidence worth remembering that the total exposure time per
%position over 10 years (in all bands) is equal to about one observing night.
\begin{itemize}
\item 90\% of time  spent on  uniform survey: every 3-4 nights, the whole observable sky scanned twice per night
\item	~100 PB of data: about a billion 16 Mpix images, enabling measurements\\ {\color{cyan} for 40 billion objects! }
\end{itemize}
{\tiny see also \url{http://www.lsst.org} and \cite{2008arXiv0805.2366I}-arXiv:0805.2366}

\column{0.55\textwidth}
	 \includegraphics[width=0.9\textwidth]{images/coverage}\\
\vspace {-5pt}
{\bf 10-year simulation of LSST survey: number of visits in u,g,r band (Aitoff projection of eq. coordinates) }\\
\end{columns}

}
\frame { \frametitle{ LSST Camera }
\begin{columns}
\column{0.6\textwidth}
 \includegraphics[width=1.0\textwidth]{images/camera}
\column{0.4\textwidth}
 \\
\vspace {-3cm}
{\large \bf The largest astronomical camera:}
\begin {itemize}
\item 2800 kg
\item 3.2 Gpix
\end {itemize}
\end{columns}
}

\frame { \frametitle{ Site as imagined and in March 2019 }
\begin{columns}
\begin{column}{0.5\textwidth}
\vspace {7cm}
	\includegraphics[width=0.95\textwidth,trim=0cm 10cm 0 10cm,clip]{images/cerroRender}
\end{column}
\begin{column}{0.5\textwidth}
\\
	\includegraphics[width=0.95\textwidth,trim=0cm 0cm 0cm 0cm,clip]{images/cerroSep2019}
\end{column}
\end{columns}

}

\frame {\frametitle{Data Management Mission Statement}
\begin{columns}
	\column{0.65\textwidth}
	      \includegraphics[width=1.0\textwidth]{images/DmMap}\\
	      \column{0.35\textwidth}
	       \\

	       DM's mission:\\
	           \textit{Stand up operable, maintainable, quality services to deliver high-quality LSST data products for science, all on time and within reasonable cost.}\\
		       \vspace{10pt}
		       Development is distributed across the Americas.\\

		       { Plus we have partners like IN2P3 (France).}\\
		       \vspace{5pt}
		       About 100 individuals $\approx$80 FTE

\end{columns}
}

\frame [allowframebreaks]{\frametitle{  Communication is difficult and important }
Have a code of conduct.
\begin{itemize}
  \item Internal
    \begin{itemize}
	    \item Data Management Leadership meeting every Monday (30 minutes) - 4 times a year longer 2.5 days (2 now virtual with moderators and 5 hour days to match time zones).
             \item Gaia DPAC executive similar (2 times physical meeting per year) - technical focus.
	    \item Both Rubin and DPAC have {\color{green} newsletters } excellent and (DPAC) well contributed to
      \item (Few) Focused working groups and working meetings
      \item DPAC First consortium meeting Nov 2015; {\color{red} 10 years perhaps late .}
      \item Rubin have one every year {\color{red} it seems too much}
      \item Though stressful personally {\color{green} Rubin putting all software under one leader is good } - {\color{red} Done rather late } hence stress .
      \item As for any project cost of entry for new people is very high {\color{blue} --- no obvious
      solution }
    \end{itemize}
  \item ESA policy initially to reduce contact between DPAC and Astrium ( now AirbusDS who constructed Gaia)
    {\color{red} not good}
  \item External
    \begin{itemize}
	    \item Perhaps could have had a better DPAC website - Rubin is somewhat ok .. {\color{blue} would really like a git backed Pull Request driven site .. but we have Drupal..}
      \item ESA PR also not great (ok as they point out they have a tiny fraction of NASA budget)
	  \item A bit better on Rubin (also fraction of NASA budget).
      \item {\color{blue}DPAC Publication policy was dealt with very late }
      \item {\color{green}Publication policy clearly in place on Rubin }
    \end{itemize}
\end{itemize}
}



\frame {\frametitle{  DM build and deploy - already challenging }
	\vspace{-1cm}
	\begin{columns}
		\column{0.4\textwidth}
		\begin{center}
			      \includegraphics[width=1.0\textwidth]{images/DMSDeployment}\\
		      \end{center}
		      \column{0.6\textwidth}
		       \\
		       \vspace{1cm}
		       DM must build everything to get LSST products (see \url{http://ls.st/dpdd})  to the users.
		       \begin{itemize}
			       \item large data sets (20TB/night)
			       \item complex analysis
			       \item aiming for small systematics
			       \item Science Alerts in under 2 minutes .. (aiming for 1 minute)
		       \end{itemize}
		       About $1\over{2}$  million lines of code (C++/python) all open source on github\\
		       \vspace{25pt}
		       {\tiny \bf diagram K.T. Lim}
	       \end{columns}
       }



\frame{\frametitle{Kubernetes-based services with ArgoCD }

\begin{columns}
\begin{column}{0.5\textwidth}

\begin{itemize}
\item Kubernetes provides powerful container orchestration and management of resources available to services
\item ArgoCD provides a framework for deploying and monitoring Kubernetes-based services with configuration management via GitOps
\item Vault provides secrets management
\item Combined, these three allow us to have automated, reproducible deployments

\end{itemize}
\end{column}
\begin{column}{0.5\textwidth}

\includegraphics[width=1.0\textwidth]{images/ArgoCD}
	\vspace{8pt}
\hfill	\textbf{\tiny  Frossie Economou frossie@lsst.org}
\end{column}
\end{columns}

}




\frame{\frametitle{Rubin Observatory Catalog 2035 }
Astronomy catalogs tend to be highly structured, tabular, somewhat predictable access.

\begin{columns}
\begin{column}{0.45\textwidth}

\begin{itemize}
\item Data, by DR11:
\begin{itemize}
\item  ~60T rows (mostly ForcedSource)
\item  ~10PB (mostly Source + ForcedSource + Object extra)
\end{itemize}
\item  Breakdown of most significant tables (rows x cols, storage):
\begin{itemize}
\item  Object: ~47B x 330, ~100TB
\item  Object extra: ~1.5T x 7,600, ~1.2PB
\item  Source: ~9T x 50, ~5PB
\item  ForcedSource: ~50T x 6, ~2PB
\end{itemize}
\end{itemize}
\end{column}
\begin{column}{0.55\textwidth}
\begin{itemize}
\item Get an object or data for small area - <10 sec
\item Scan through billions of objects - $\approx$ 1 hour
\item Deeper analysis (Object\_*) - $\approx$ 8 hours
\item Analysis of objects close to other objects -  $\approx$ 1 hour, even if full-sky
\item Analysis that requires special grouping - $\approx$ 1 hour, even if full sky
\item Source, ForcedSource scans - $\approx$ 12 hours
\item Cross match \& anti-cross match with external catalogs - $\approx$ 1 hour
\end{itemize}
\end{column}
\end{columns}
}


\frame{\frametitle{Qserv \tiny(slides from Fritz Muller) }

\begin{columns}
\begin{column}{0.6\textwidth}
\begin{itemize}
\item Shared-nothing MPP RDBMS (SQL, throughput, horizontal scaling)
\item  Spherical partitioning with overlap (near-neighbor self-joins)
\item  Shared scans (concurrent query load)
\item  Replicated data (resiliency)
\item  Fixed-purpose, dedicated hardware (cost, predictability)
\end{itemize}
\end{column}
\begin{column}{0.4\textwidth}

\includegraphics[width=0.90\textwidth]{images/spher}
	 Tesselation see  \cite{2001misk.conf..638O}
\end{column}
\end{columns}

Design optimized for use case + hardware efficiency \citeds{LDM-135}

Built on project at SLAC, leverage existing tech within Stanford (MariaDB, MySQL Proxy,
XRootD, Google protobuf, Flask)


100\% open source  \url{https://github.com/LSST/Qserv}
}

\frame{\frametitle{Shared Nothing Massively Parallel Processing }

\begin{columns}
\begin{column}{0.45\textwidth}

\includegraphics[width=1.1\textwidth]{images/distribute-combine}\\
		Recent scale tests: \url{https://dmtr-071.lsst.io}\\
		Perf and BigQuery: \citeds{Document-31100}
\end{column}
\begin{column}{0.5\textwidth}

\begin{itemize}
\item Ultimate target platform ~300 nodes in 2 international data-centers
\item	Development cluster (CC-IN2P3):
\begin{itemize}
\item  400 cores, 800 GB memory, 500 TB storage
\item  ~100 TB synthetic dataset on 2 x 25 nodes
\end{itemize}
\item Prototype Data Access Center (NCSA):
\begin{itemize}
	\item  500 cores, 4 TB memory, 700 TB storage
	\item  ~100 TB science dataset (SDSS Stripe 82 +
		WISE) on 30 nodes
	\item  + HSC reprocessing + GAIA DR2 coming up
\end{itemize}
\end{itemize}
\end{column}
\end{columns}
}




\frame{\frametitle{Guidelines,  tools,  conventions }
\begin{itemize}
    \item {\color{green} Its great having extensive guidelines - it was also something super on Gaia}
   \item DPAC had full engineering Guide from early on \citellp{LL:WOM-011} - how to Mantis, SVN, Java guide \ldots
   \item \url{http://developer.lsst.io} is a full developer guide - everything from git commit messages to style guides for Python and C+.

   \item \url{http://pipelines.lsst.io} documents the main software release(s)
	\begin{itemize}
	\item worst of both worlds - it is a monolithic release  - but made up of > 120 git repos
	\item Still using \emph{in house} tools like EUPS \url{https://github.com/RobertLuptonTheGood/eups}
	\item moving toward conda-forge
	\item {\color{green} All open source (GPL) on github.com}
	\end{itemize}
\item  Language (spoken/written) and conventions are also super important
    \begin{itemize}
	    \item Single language projects  (like US or UK) fall more easily in the trap of \emph{believing} they speak  about the same topics because they speak language X.
	    \item {\color{blue} RubinObs lacks something like \citeds{LL:BAS-003}}
    \end{itemize}

\end{itemize}
}


\frame {\frametitle{  Software (and data) licensing }
\begin{itemize}
  \item Protect intellectual property  --- grant use to the consortium
  \item Often forgotten or not well dealt with -  or worse ignored!
\end{itemize}
\vspace{-0.2cm}
\begin{columns}
\column{0.4\textwidth}
\begin{itemize}
  \item DPAC agreed to LGPL \citellp{LL:WOM-019} -  some institutes e.g. ESA, do not allow GPL code
  \item LGPL in ESA involved lawyers and directors and time  {\color{red} Now ESA have own open license.}

\item Rubin GPL .. would prefer APL.
\item {\color{blue} Gaia Data license only after DR1 } (open with attribution).
\end{itemize}
\column{0.6\textwidth}
\begin{center}
\vspace{-0.2cm}
      \includegraphics[width=1.0\textwidth]{dilbertsl}\\
{\tiny You may use up to seven (7) cartoons per year at no costs as part of our fair use policy.}
\end{center}
\end{columns}

\vspace{5pt }
	 Have only mentioned Data license  on Rubin
}





\frame {\frametitle{SDSS image }
	\begin{columns}
		\column{0.45\textwidth}
	      \includegraphics[width=1.0\textwidth]{images/SDSScosmos}\\
		      \column{0.55\textwidth}
		      \\
		       \vspace{1cm}
	       Nice colors \cite{2004PASP..116..133L}\\
	       $\approx  3.5 \arcmin$\\
		       \vspace{20mm}
	       {\tiny \bf{Image  Robert Lupton}}
	       \end{columns}
}
\frame {\frametitle{Hyper Suprime Cam (HSC) on Subaru }
\begin{columns}
      \column{0.45\textwidth}
%\vspace{-20pt}
	      \includegraphics[width=1.0\textwidth]{images/HSCcosmos}\\
      \column{0.55\textwidth}
		       \\
		       HSC image (COSMOS) from g,r(1.5 hrs) ,i(3 hrs) PSF matched co-add ($\approx 27.5$)\\
		       Challenges:

\begin{itemize}
\item Unknown statistical distributions,   Truncated, censored and missing data, Unreliable quantities (e.g. unknown systematics and random errors)
\item PSF - short exposure - atmosphere dominated ?

\item Photometry  challenging - will Gaia help ..
\item Everything is blended!!
\end{itemize}
		{\tiny       Processed with  {\em LSST Stack} \url{https://pipelines.lsst.io/}\\
	        \bf{Image HSC collaboration,  Robert Lupton}}
\end{columns}
       }



\frame{\frametitle{Catalog extraction }
\begin{columns}
      \column{0.45\textwidth}
		       \vspace{-20pt}
	      \includegraphics[width=1.0\textwidth]{images/square/jupyterlab_nb}\\
      \column{0.55\textwidth}

\begin{itemize}
\item Identifying sources and disentangling them becomes more difficult as we have deeper images.
\item Left our typical Jupyter setup runs

\begin{itemize}
\item  Insrtrument signature removeval
\item  calibration
\item  source extraction
\item  overlay extracted information on cleaned image
\end{itemize}
\item This helps users understand how catalogs are produced.
\end{itemize}
\end{columns}
}


