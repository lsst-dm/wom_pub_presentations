\frame{\frametitle{ The Era Of Surveys and Archives }
\begin{columns}
\column{0.6\textwidth}
\includegraphics[width=1.0\textwidth,trim=0 0 0 0] {images/AstronomerCartoon}\\
\column{0.4\textwidth}
\\
\vspace{-0.2cm}
\includegraphics[width=1.0\textwidth,trim=0 0 0 0] {images/HSTpubs}\\
{\tiny \url{https://archive.stsci.edu/hst/bibliography/pubstat.html}}\\

{\small \ldots indicates  archival  research  probably play an important role in the scientific success of XMM-Newton
\cite{2014AN....335..210N}}

\end{columns}
}


\frame{\frametitle{ Gaia Archive}
    \begin{tikzpicture}
	  \node (viz) {
  \includegraphics[width=0.7\textwidth,trim=0 0 0 0] {gaiaim/gaiaviz}
	  };
	  \node(a) [right=15mm of viz] {};
	  \node (tap)[below=-20mm of a] {
\includegraphics[width=0.4\textwidth,trim=0 0 0 0] {gaiaim/gaiatap}
	  };

    \end{tikzpicture}

\hspace{10pt}
All Gaia data is publicly accessible at \url{https://gea.esac.esa.int/archive/}
}


\frame{\frametitle{ LSST Science Platform}
\begin{columns}
\column{0.6\textwidth}
\includegraphics[width=1.1\textwidth,trim=0 0 0 0] {fig-lsst-science-platform-extended} \\
\column{0.4\textwidth}
\vspace{-0.3cm}
\\
For DR2:
\begin{itemize}
\item Computing:2,400 cores ($\approx 18$ TFLOPs)
\item File storage: $\approx 4 $PB  (VOSpace)
\item Database storage: $\approx 3 $PB (MYDB)

\end{itemize}
\end{columns}
\vspace{5pt}
The Science Platform has three user facing aspects: the Portal (novice), the JupyterLab (intermediate), and the Web APIs (expert and remote tools).\\
Vision: \citeds{LSE-319} --- Design: \citeds{LDM-542} --- Test: \citeds{DMTR-51}
\vspace{5pt}

This is the sort of environment users now \emph{expect} to have - it is no longer novel. We are finally
  {\color{red} Bringing code to the data} - almost did with GAVIP \cite{2016SPIE.9913E..1VV}

}
