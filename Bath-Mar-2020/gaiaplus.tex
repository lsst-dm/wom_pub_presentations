
\frame {\frametitle{ Parameters and data models }
\begin{itemize}
%\item Which value do we use for speed of light, measurements of Gaia instrument
  \item Avoid different values of constants in peoples code \ldots
  \item The Gaia Parameter Database was set up early on for this \citep{2005ESASP.576...67D}
    \begin{itemize}
      \item all constants in one place; web searchable configuration controlled (Only updated by Jos De Bruijne)
      \item published as constants for Java (can also do C, Fortran\ldots) so you may refer to a particular version
    \end{itemize}
  \item then the actual data model --- what exactly is an AstroElementary?
    \begin{itemize}
      \item entire data model defined in multi-user dictionary tool; includes Units on each field.
        \begin{itemize}
          \item good for astronomers --- computer people find it harder to handle
        \end{itemize}
      \item from it we generate data instance classes and schemas for storage.
      \item {\color{green} ONLY data model not processing} --- all objects are dumb (had discussion with KT  and Mario on this)
	\item This was in UML (Rose) in the 90s but it was impractical to continue..
	\item Rubin Obs also had Rational Rose and still have Magic Draw .. but the Data model is in python now more or less.
    \end{itemize}
  \item {\color{green} These are logical extensions of having agreed conventions\ldots}
\end{itemize}
}


\frame {
  \frametitle{ Inevitably we must document ..  }

\begin{columns}
	\column{0.5\textwidth}
\begin{center}
   \includegraphics[width=0.9\textwidth,trim=0cm 0cm 0cm 0cm]{images/fops}\\
\end{center}
	\column{0.5\textwidth}
 \\
\vspace{1cm}
Gaia Flight Operations Procedures (FOP) paper copy in case the computers fail - could be useful!
\begin{center}
\vspace{5pt}
{\color{red} But we should avoid {\em write only} documents.}
\end{center}
\end{columns}
}




\frame {\frametitle{ Have a standard: DPAC follows ECSS, Rubin MBSE }
\begin{columns}
  \begin{column}{0.5\textwidth}
%\pgfputat{\pgfxy(0,0)}{\pgfbox[left,top]{
    %\begin{center}
      \includegraphics[width=1.0\textwidth]{DocTree}\\
    %\end{center}
	  {\tiny Rubin Obs DM  Doctree (\citeds{LDM-294})}\hfill
	  Rubin less rigorous than DPAC - but then its on the ground
%}}
  \end{column}
  \begin{column}{0.5\textwidth}
    \textbf{\tiny European Cooperation for Space Standardization}
    \begin{itemize}
      \item Standards need to be tailored
        \begin{itemize}
          \item LaTeX Templates/examples provided
          \item Documents are iterated ---  All of this is done for all  products.
          \item {\color{green} It is very good to have a standard set of documents augmented by technical notes and streamlined }
          \item Some still found it too heavy --- other reports requested beyond the standard ones.
        \end{itemize}
\item DPAC had sufficient QA people ($\sim1/$CU) from the start (Rubin NONE)
    \end{itemize}
  \end{column}
\end{columns}
}



\frame{\frametitle{Development tools}
\vspace{-8pt}
\begin{itemize}
  \item All DPAC code and docs in Subversion, Rubin in GitHub.
    \begin{itemize}
      \item Access control according to Group membership in the LDAP
    \end{itemize}
  \item Centralized issue tracking (includes risks and actions)
    \begin{itemize}
      \item DPAC - Mantis eventually Jira.  Rubin  Jira
  \end{itemize}
\pause
  \item{\color{green} Having one language is good \citep{2011arXiv1108.0355O}} Gaia agreed on Java 2006, Rubin Python/C++ (2009  perhaps earlier)

    \begin{itemize}
      \item Can have a library of standard  routines  GaiaTools (Relativity, Field Angle Calculator,
        Ephemeris handling\ldots)
        \begin{itemize}
          \item {\color{red} The set of routines were not defined hence GaiaTools is a bit of
          hodgepodge mess\ldots}
          \item {\color{blue} Counter argument for common tools is (unnecessary)
          interdependence\ldots} we have that on Rubin
        \end{itemize}
      \item {\color{blue} virtual machines make some reasons for Java invalid}
      \item Rubin builds take a long time ..

    \end{itemize}
\end{itemize}
}



