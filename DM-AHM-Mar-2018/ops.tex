
\section{LSST Operations}
\frame{\frametitle{LSST operations proposal }
\begin{itemize}
\item Proposal was submitted in summer to NSF/DOE to fund LSST operations
\item A joint agency review was completed with positive feedback Dec 7$^{th}$ 2017.
\item LSST Operations key points:
\begin{itemize}
\item Distributed over SLAC, Tucson, La Serena, NCSA Illinois

\item LSST  Data Facility element
\item To fit in the new National Center for Optical and infrared Astronomy (NCOA) framework
\end{itemize}
\end{itemize}

{\large \color{red} What I will present  are elements of the proposal - it is not accepted yet.}


}

\frame{\frametitle{LSST Operations - Distributed}

\begin{columns}
\column{0.2\textwidth}
\vspace {-6cm}
\\
100 - 200\,Gbps international links\\
\vspace{5pt}
40 - 200\,Gbps summit base\\
\vspace{5pt}
See \citedsp{LSE-78}\\
\vspace{15pt}
{\tiny \bf Jeff Kantor }

\column{0.8\textwidth}
\includegraphics[width=1\textwidth]{images/SitesDataflow}\\
\hfill {\tiny \bf Emily Acosta }
\end{columns}
}



\frame{\frametitle{LSST Operations - Organisation}
\includegraphics[width=0.9\textwidth]{images/LSSTopsHighLevelOrg}
{\tiny \bf Beth Willman }
}
\frame{\frametitle{LSST Operations - Communications}

\begin{columns}
\column{0.6\textwidth}
\includegraphics[width=0.9\textwidth]{images/LSSTopsCom}\\
{\tiny \bf Phil Marshall }
\column{0.4\textwidth}
\vspace{-4cm}
\begin{itemize}
\item Formal and informal  channels in place
\item Regular weekly daily meetings as well as Jira for tracking
\item Slack, community.lsst.org, etc as well
\end{itemize}
\end{columns}
}

\frame{\frametitle{Observatory Operations Activities (on summit) }
\begin{columns}
\column{0.44\textwidth}
\includegraphics[width=1\textwidth]{images/summit24hrs}\\
\column{0.56\textwidth}
\vspace{-6cm}
\\
High level 24 hour activity (50FTE):
\begin{itemize}
\item Regular maintenance
\item Evening calibrations
\item Nightly observations
\item Day crew, night crew shift 1, night crew shift 2
\item Software
\item ITC supports daily data transmission
\end{itemize}
\vspace {1cm}
\hfill {\tiny \bf Chuck Claver }
\end{columns}
}


\frame{\frametitle{What is Science Operations ?}
\vspace{-0.2cm}
\center {\bf Deliver the science products defined  \citeds{LSE-163} to the community.}
\begin{columns}
\column{0.5\textwidth}
\vspace{-5.7cm}
\\That entails working with :
\begin{enumerate}
\item Chilean Operations to make sure the observations are scientifically good
\item the Data Facility to ensure production is running correctly
\item Survey performance to make sure over the longer period we will meet science goals.
\item EPO to communicate our successes {\bf and possible more importantly our short comings}.
\end{enumerate}
\column{0.5\textwidth}
  \includegraphics[width=1.0\textwidth]{images/LSSTopsCom}\\
\end{columns}
{\bf \color{red} AND  Maintain/improve software systems bought over from data management}
}


\frame{\frametitle{Science Operations: Curating LSST Science }
On a daily basis Science Operations Staff are looking at data quality from both instrument and software perspectives asking many questions (28FTE):

\begin{itemize}
\item Are the alerts as good as we can make them ?
\item Are there any data products we could deliver/improve?
\item Are the changes we should request on the telescope ?
\item Was there some event (weather/hardware) affecting data we should be telling the community about ?

\end{itemize}
Longer term :
\begin{itemize}
\item Are there any disturbing trends in Key Performance Metrics?
\item How is the Data Release Product quality ?
\end{itemize}
}

\frame{\frametitle{Organization: Science Operations reporting}
\begin{center}
  \includegraphics[width=0.7\textwidth]{images/LSSTopsHighLevelOrg}\\
\end{center}
\vspace{-1cm}
\begin{itemize}
\item Science operations is one of the main pillars of LSST operations
\item FY23 estimate is to have 28FTE in the department
\item going down to  23FTE over 4 years
\end{itemize}
}

\frame{\frametitle{Organization: Science Operations groups}
We are organized into three groups, which parallel the key components of science deliverables:\\

\begin{columns}
\column{0.3\textwidth}
\begin{itemize}
\item  Observatory Science
\item  Science Algorithms and Pipelines
\item Science Platform.
\end{itemize}


\column{0.7\textwidth}
\vspace{-1cm}
\begin{center}
  \includegraphics[width=0.9\textwidth]{images/LSSTSciOpsOrg}\\
\end{center}

\end{columns}
}


\frame{\frametitle{Observatory Science Group}
5.5 FTE starting in Chile to learn the instrument and them migrating to Tucson\\
This group should:
\begin{itemize}
\item   Understand end-to-end impact of hardware and summit conditions on science
\item   Assure  science images can deliver to LSST’s science requirements
\item   Track and document hardware issues
\item   Propose changes to the telescope, instrumentation or software for efficiency

\end{itemize}
}



\frame{\frametitle{ Algorithms and Pipelines Group }
8 FTE in various institutes eventually converging on Tucson.
\begin{itemize}
\item   Tracking and documenting the data product quality for Alert Production using SDQA and other tools;
\item Proposing when changes need to be made to all aspects of the Alert Production algorithms and pipelines (including distribution of alerts and orbits for solar system objects);
\item Proposing when changes need to be made to all aspects of the annual data release algorithms (including calibration, Multifit, and deblender code); and
\item Proposing to accept or reject software changes based on a scientific validation of new algorithms and an understanding of their impact on required computational resources.

\end{itemize}
}

\frame{\frametitle{Science Platform }
4.5 FTE in Tucson
\begin{itemize}
\item  maintain and evolve the Science User Interface

\item maintain and evolve User Services

\end{itemize}
"Science Platform” refers to the functions that will enable the community to conduct analyses and generate new data products, near the data, and will include Jupyter widgets to enable users to use Jupyter notebook with SUI visualization and a set of special visualization templates for interactive Quality Analysis (QA).
}

\frame{\frametitle{Science Operations Staffing }

\begin{columns}
\begin{column}{0.4\textwidth}
28 FTE starting operations going down to 23 FTE over 4 years.\\
\vspace{15pt}
Combination NCOA and SLAC positions.\\
\vspace{15pt}

Initially support also in institutes e.g. Princeton   \\
\vspace{15pt}

\end{column}
\begin{column}{0.6\textwidth}

\vspace{-10pt}
\includegraphics[width=\textwidth]{SciOpsFTE}
\end{column}
\end{columns}

}


\frame{\frametitle{LSST Data Facility Organisation }
Organized to provide cohesive, reliable, and efficient production services.
\begin{columns}
\column{0.4\textwidth}
\begin{itemize}
\item Scientific Production Services and Data, Compute and IT Security Services provide the application-level elements of LDF services.
\item Production Service Software maintains and evolves the LDF software.
\item ITC and Facility Operations provide foundational hardware support, building on general facility staff.

\end{itemize}


\column{0.6\textwidth}
\vspace{-1cm}
\begin{center}
  \includegraphics[width=0.9\textwidth]{LDForg}\\
\end{center}

\end{columns}
}


\frame{\frametitle{LDF responsibilities }
\begin{itemize}
\item Ingest data from all Observatory cameras and the EFD.
\item Process data promptly for alert generation and in batch for daily to yearly calibrations and data releases.
\item Provide alerts for transient sources to event brokers for distribution.
\item Oversee and manage the hardware and data archive, including data integrity and disaster recovery, at both NCSA and the Chilean DAC.
\item Monitor and coordinate the wide-area network infrastructure.
\item Provide infrastructure support for data access services to the user community.
\item Provide central Authentication and Authorization (AA) security-related services, including access control for science data rights holders and network security controls for the operational facilities.

\end{itemize}
}

\frame{\frametitle{Scientific Production Services Group (6.75 FTE)}

Ensures the reliable and timely generation of prompt, nightly, and annual data products.


\begin{itemize}
\item Provides first-order tactical scientific data quality assurance (SDQA), focused on an initial diagnosis of issues during processing.
\item Communicates operational feedback to Science Operations on status of data products production.
\item Receives pipeline codes from Science Operations and verifies that processing is operationally feasible at production scale.
\end{itemize}
Staffed by Lead Production Scientist (1.0 FTE), Production Scientists (3.0 FTE), and Computational Facility Scientists (2.75 FTE).
}



\frame{\frametitle{Data, Compute, and IT Security Services Group (8.5 FTE) }
Provides operational support for foundational services.
\begin{itemize}
\item File management services on the LSST data backbone including ingestion, distribution, migration, replication, deletion, backup, disaster recovery.
\item Managed database administration.
\item Batch computing and containerized application management.
\item Authentication and Authorization (AA) and network security monitoring across LSST.
\end{itemize}
Staffed by Foundational Services Manager (0.75 FTE), File Services Administrators (1.75 FTE), Database Administrators (3.75 FTE), Workflow Services Administrators (0.75 FTE), Operational Security Engineers (1.5 FTE).

}

\frame{\frametitle{Production Service Software Group (4.3 FTE) }
Maintains, enhances, and evolves supporting software infrastructure required for production, including
\begin{itemize}
\item Qserv, the large-scale custom scientific database.
\item Image acquisition and archiving.
\item Near real-time execution orchestration, batch processing, data staging.
\item LSST data backbone management.
\item Data access, bulk export, hosting the LSST Science Platform.
\item General LDF operational and service management software.

\end{itemize}

Staffed by Lead Software Engineer (1.0 FTE), Service Software Engineers (3.25 FTE), and Business Process Programmer (0.05 FTE).

}


\frame{\frametitle{ITC and Facility Operations Group (12.25 FTE) }
Provides system administration and physical system operations of the LSST Data Facility computing, storage, and communications infrastructure.

\begin{itemize}
\item Operation of systems at NCSA and Base Site, including 24x7 support.
\item Deployment and decommissioning of ITC infrastructure, physical space planning, coordination with Base and satellite facilities.
\item Operation, management and support of Chilean and U.S. national and international networks for LSST usage.

\end{itemize}
Staffed by IT Center Systems Manager (1.0 FTE), IT System Administrators (3.25 FTE), IT System Technician (1.0 FTE), Computing Facility Operator (0.5 FTE), Facility Network Engineers (1.75 FTE), Storage Engineers (4 FTE), WAN Technical Manager (0.5 FTE), and WAN Architect (0.25 FTE).

}

\frame{\frametitle{ LSST Data Facility}
\vspace{10pt}
\begin{columns}
\begin{column}{0.4\textwidth}
\vspace{-5.5cm}
\\
NCSA scientific and technical staff drawn from existing core areas of Center expertise. \\
\vspace{15pt}
FNAL due to proximity and existing collaborative relationships (e.g., DES). \\
\vspace{15pt}
SLAC due to intimate knowledge of software created during the LSST construction project. Currently includes QSERV\\
\end{column}
\begin{column}{0.6\textwidth}
\includegraphics[width=\textwidth]{LDFFTE}\\
\end{column}
\end{columns}
}

\frame{\frametitle{LSST Operations - Proposed Staffing profile}
\includegraphics[width=0.9\textwidth]{images/opsStaffProfile}
}


\section{Commissioning}
\frame {\frametitle{Hopefully {\color{red}not} DM commissioning }
	\vspace{-9pt}
\begin{center}
\includegraphics[width=0.9\textwidth]{images/Raft_of_the_Medusa_-_Theodore_Gericault} \\
	\vspace{-3pt}
{\tiny \bf Raft of the Medusa, Theodore Gericault}
\end{center}
}

\frame{\frametitle{Operations Rehearsals }
\tiny
\input {orstab}
\normalsize
}

\frame{\frametitle{LSST Operations - Construction transition}
\includegraphics[width=1.0\textwidth]{opsTransFTE}
}


\frame{\frametitle{DM Transition }
\begin{itemize}
\item DM --> Science Operations and LSST Data Facility
\item Starting to look at specific roles and the transition of DM staff into them
\item  As you see in the previous chart there will be a problem in commissioning to find enough FTEs
\item We may have to extend DM constructing further
\item A few DM people will  be in in Chile for some period of commissioning (Jim, Merlin, Robert \ldots)

\begin{itemize}
\item based on need to see the  hardware
\item  We are exploring this in the commissioning rehearsals
\item Most of us can work from our home institutes or perhaps congregate in Illinois or Tucson for specific events.
\end{itemize}
\end{itemize}
}

