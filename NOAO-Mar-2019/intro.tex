

\beamersetaveragebackground{black}

\setlength\myleftmargin{0cm}

\thispagestyle{empty}
\frame[plain]{
 \centering
 \vspace{-25pt}
    \includegraphics[scale=0.47]{images/tinto}

}

\thispagestyle{empty}
\frame[plain]{
\frametitle{\color{yellow} Milky way analogue (a better model)}
\vspace{-3.3cm}\hspace{-10pt}\includegraphics[scale=0.4] {images/ngc1232}
}
\thispagestyle{empty}
\frame[plain]{
\frametitle{\color{yellow} Milky way analogue}
\vspace{-3.3cm}\hspace{-10pt}\includegraphics[scale=0.4] {images/ngc1232}\\
\vspace {-7cm}
{\color{yellow}
%\rowcolors{1}{blue!20}{blue!20}
\begin{tabular}{p{\textwidth}}
{\huge Milky way analogue}\\
\\
\\
\\
\\
\\
 \hspace{30pt}{\huge `Our sun'$\rightarrow$ \normalsize }\\

{ Our view is severely obstructed by the dust in the disk and relatively little is known about the origin, history, and structure of our own Galaxy}\\
\end{tabular}
\vspace{-4cm}
\begin{tabular} {r}
\\
 {We are unraveling the formation, composition, and evolution of the Galaxy}\\

\hspace{11cm} {\tiny Jos de Bruijne \normalsize }
\end{tabular}
}}

\frame{\frametitle{Another Milky way Analogue (NGC4565)}
\centering
\includegraphics[width=0.65\textwidth]{ngc4565}
%NGC 4565, ngc4565_mclaughlin_big.jpg
%Needle Galaxy in Coma, William McLaughlin, 50 million light years
}

\setlength\myleftmargin{0.5cm}

\thispagestyle{empty}
\frame[plain]{
\frametitle{\color{yellow} Stellar motions}
\color{yellow}
\setbeamercovered{invisible}
\begin{tabular}{p{1.0\textwidth}}

Stellar motions can be predicted into the future and calculated for times in the past when all 6 phase-space coordinates (3 positions, 3 velocities) are known for each star
\\ \pause
The evolutionary history of the Galaxy is recorded mainly in the halo, where incoming galaxies got stripped by our Galaxy and incorporated
\\
\vspace{-0.7cm}
\begin{center}
\includegraphics[scale=0.55,trim=0 2.2cm 0 0] {merger}
\end{center}

\\ \pause
In such processes, stars got spread over the whole sky but their energy and (angular) momenta were conserved. Thus, it is possible to work out, even now, which stars belong to which merger and to reconstruct the accretion history of the halo
{\tiny (de Bruijne)}
\end{tabular}

}




\frame[plain]{\frametitle{Origin of the Milky Way }
\vspace{-15pt}
\begin{center}
%only seems to work wth avi in same directory as pdf ..
   % \movie[width=9cm]{\includegraphics[width=9cm]{Aq-Amina-xz_100kpc_n-frame04sec}}{Aq.mov}
    \href {file:/Aq.mov} { \includegraphics[width=9cm]{Aq-Amina-xz_100kpc_n-frame04sec}}
\end{center}
\vspace{-15pt}
  {\scriptsize \href{run:Aq.mov}{Movie credit: Amina Helmi, University of Groningen}\hfill slides from Anthony Brown, Leiden University}

}
%\setbeamertemplate{frametitle}{{\centering \logoleft \hfill \headercenter \hfill \logoright}}

\frame{\frametitle{Origin of the Milky Way }
\vspace{-15pt}
\begin{center}
    \includegraphics[height=7cm]{Aq-Amina-xz_100kpc_n-frame50sec.png}
	\hspace{1cm}
    \includegraphics[height=7cm]{ngc5907_small.jpg}
  \end{center}
  \vskip-1.0cm
  {\color{yellow} \hspace{11cm}\scriptsize Image credit: R.\ Jay GaBany}
  \vspace{-1.0cm}
  {\scriptsize Just published using DR2 \cite{2041-8205-860-1-L11}}
  }



\thispagestyle{empty}
\beamersetaveragebackground{black}
\frame[plain]{ \frametitle{Beyond our galaxy}
	\vspace{-13pt}
\includegraphics[width=\textwidth,trim=0 0 0 0] {images/planckhistory}\\
	\vspace{-08pt}
\begin{columns}
\column{0.38\textwidth}
\includegraphics[width=1.0\textwidth,trim=0 0 0 0] {images/DarkMatterPie}\\
\column{0.62\textwidth}
\\
\vspace{-60pt}
{\color{yellow} The modern cosmological models can explain all observations, but need to {\em postulate} dark matter and dark energy (though gravity model could be wrong, too) }
\end{columns}
}
\beamersetaveragebackground{white}


\frame{\frametitle{The Time Domain}
\begin{columns}
\begin{column}{0.6\textwidth}
LSST scans the sky repeatedly and compares the images with templates allowing us to:
\begin{itemize}
\item discover new, distant transient events
\item study variable objects in universe
\item capture rare and exotic objects
\item  gain new insight into known transients:
\begin{itemize}
\item  new remnants of dead massive stars, including neutron star and black hole binaries;
\item  variability at the heart of distant galaxies - feeding habits of the rapacious supermassive black holes at their centers;
\item  catch the faint cosmological explosions of dying, merging stars, illuminating where the universe’s heavy metals are forged.
\end{itemize}
\end{itemize}
\end{column}
\begin{column}{0.4\textwidth}

\vspace{-18pt}
\includegraphics[width=1\textwidth]{900px-Crab_Nebula}
{\small The Crab Nebula, result of a supernova noted in 1054 A.D.}\\
{\tiny \bf By NASA, ESA, J. Hester and A. Loll (Arizona State University) - Hubble}
\end{column}
\end{columns}




}





\frame{\frametitle{{\color{red} Killer asteroids}: the impact probability is not 0! }
\begin{columns}
\column{0.5\textwidth}
\includegraphics[width=1.0\textwidth,trim=0 0 0 0] {images/asteroidImpacts}\\

\column{0.5\textwidth}
\\
\vspace{-7cm}
{\bf LSST is the only survey capable of delivering completeness specified in the 2005 USA Congressional NEO mandate to NASA (to find 90\% NEOs larger than 140m)}\\
\vspace{1cm}
\includegraphics[width=1.0\textwidth,trim=0 0 0 0] {images/BarringerCrater}\\
\vspace{3mm}
The Barringer Crater, Arizona:    a 40m object 50,000 yr. ago

\end{columns}
}


\frame{\frametitle{ The Era Of Surveys }
\begin{columns}
\column{0.65\textwidth}
\includegraphics[width=1.0\textwidth,trim=0 0 0 0] {images/AstronomerCartoon}\\
\column{0.35\textwidth}
\\
\vspace{-7.2cm}
\includegraphics[width=1.0\textwidth,trim=0 0 0 0] {images/HSTpubs}\\
{\tiny \url{https://archive.stsci.edu/hst/bibliography/pubstat.html}}\\

{\small \ldots indicates  archival  research  probably play an important role in the scientific success of XMM-Newton
\cite{2014AN....335..210N}}

\end{columns}
}



