
\frame{\frametitle{The Time Domain}
\begin{columns}
\begin{column}{0.6\textwidth}
LSST scans the sky repeatedly and compares the images with templates allowing us to:
\begin{itemize}
\item discover new, distant transient events
\item study variable objects in universe
\item capture rare and exotic objects
\item  gain new insight into known transients:
\begin{itemize}
\item  new remnants of dead massive stars, including neutron star and black hole binaries;
\item  variability at the heart of distant galaxies - feeding habits of the rapacious supermassive black holes at their centers;
\item  catch the faint cosmological explosions of dying, merging stars, illuminating where the universe’s heavy metals are forged.
\end{itemize}
\end{itemize}
\end{column}
\begin{column}{0.4\textwidth}

\vspace{-18pt}
\includegraphics[width=1\textwidth]{900px-Crab_Nebula}
{\small The Crab Nebula, result of a supernova noted in 1054 A.D.}\\
{\tiny \bf By NASA, ESA, J. Hester and A. Loll (Arizona State University) - Hubble}
\end{column}
\end{columns}




}


