\frame { \frametitle{ LSST:uniform sky survey }
\begin{columns}
\column{0.45\textwidth}
\vspace {-5.5cm}
 \\
An optical/near-IR survey of half the sky in ugrizy bands to r~27.5 (36 nJy) based on 825 visits over  a 10-year period: {\em deep wide fast}.
%It’s about 5,000 sq. deg. per night, *twice*, on
%average. That is, about 1,000 visits per night on average. You return to
%the same position on the sky in about 3-4 nights (in any band). Btw, it’s
%a nice coincidence worth remembering that the total exposure time per
%position over 10 years (in all bands) is equal to about one observing night.
\begin{itemize}
\item 90\% of time  spent on  uniform survey: every 3-4 nights, the whole observable sky scanned twice per night
\item	~100 PB of data: about a billion 16 Mpix images, enabling measurements\\ {\color{cyan} for 40 billion objects! }
\end{itemize}
{\tiny see also \url{http://www.lsst.org} and \cite{2008arXiv0805.2366I}-arXiv:0805.2366}

\column{0.55\textwidth}
	 \includegraphics[width=0.9\textwidth]{images/coverage}\\
\vspace {-5pt}
{\bf 10-year simulation of LSST survey: number of visits in u,g,r band (Aitoff projection of eq. coordinates) }\\
\end{columns}

}
\frame { \frametitle{ LSST Camera }
\begin{columns}
\column{0.65\textwidth}
 \includegraphics[width=1.0\textwidth]{images/camera}
\column{0.35\textwidth}
 \\
\vspace {-5cm}
{\large \bf The largest astronomical camera:}
\begin {itemize}
\item 2800 kg
\item 3.2 Gpix
\end {itemize}
\end{columns}
}

\frame { \frametitle{ Site as imagined and in March 2019 }
\begin{columns}
\begin{column}{0.5\textwidth}
	\includegraphics[width=0.95\textwidth,trim=0cm 10cm 0 10cm,clip]{images/cerroRender}
\end{column}
\begin{column}{0.5\textwidth}
\\
\vspace{+2cm}
	\includegraphics[width=0.95\textwidth,trim=0cm 0cm 5cm 0cm,clip]{images/cerroMar2019}
\end{column}
\end{columns}

}

\frame {\frametitle{  DM build and deploy - already challenging }
	\vspace{-1cm}
	\begin{columns}
		\column{0.4\textwidth}
		\begin{center}
			      \includegraphics[width=1.0\textwidth]{images/DMSDeployment}\\
		      \end{center}
		      \column{0.6\textwidth}
		       \\
		       \vspace{1cm}
		       DM must build everything to get LSST products (see \url{http://ls.st/dpdd})  to the users.
		       \begin{itemize}
			       \item large data sets (20TB/night)
			       \item complex analysis
			       \item aiming for small systematics
			       \item Science Alerts in under 2 minutes .. (aiming for 1 minute)
		       \end{itemize}
		       About $1\over{2}$  million lines of code (C++/python) all open source on github\\
		       \vspace{25pt}
		       {\tiny \bf diagram K.T. Lim}
	       \end{columns}
       }



\frame {\frametitle{  Data Backbone}
\begin{columns}
\column{0.45\textwidth}
      \includegraphics[width=1.0\textwidth]{images/DataBackbone}\\
\column{0.55\textwidth}
 \\
 \vspace{-5cm}
 One small box on the previous slide was Data Backbone.\\ That hides several things \\
 \begin{itemize}
	 \item Qserv - the LSST end user database
 \begin{itemize}
	 \item{\color{red} Custom Massive Parallel Processing (MPP) }
	 \item allow queries on $\approx 20$ Petabytes of tabular data
	\item $4 \times 10^{10}$ objects, $4 \times 10^{13}$ sources (observations)
 \end{itemize}
\item All the networks : we now have fiber to the mountain and from La Serena to NCSA (two routes)
 \end{itemize}
\vspace{25pt}
{\tiny \bf diagram K.T. Lim}
\end{columns}
}

\frame {\frametitle{Data flow }
\begin{columns}
\column{0.65\textwidth}
      \includegraphics[width=1.0\textwidth]{images/NearRealTimeDataFlow}\\
\column{0.35\textwidth}
\\
\vspace{-6cm}
Lots to do every night ..\\
Plus annually there is a data release \\
      \includegraphics[width=1.0\textwidth]{images/AnnualReprocessingDataFlow}\\

      {\tiny \bf Images from K.T. Lim}
\end{columns}
}

\frame {\frametitle{SDSS image }
	\vspace{-1mm}
	\begin{columns}
		\column{0.5\textwidth}
		\vspace{-2mm}
	      \includegraphics[width=1.0\textwidth]{images/SDSScosmos}\\
		      \column{0.5\textwidth}
		      \\
		       \vspace{1cm}
	       Nice colors \cite{2004PASP..116..133L}\\
	       $\approx  3.5 \arcmin$\\
		       \vspace{20mm}
	       {\tiny \bf{Image  Robert Lupton}}
	       \end{columns}
}
\frame {\frametitle{Hyper Suprime Cam (HSC) on Subaru }
	\vspace{-1mm}
	\begin{columns}
		\column{0.5\textwidth}
		\vspace{-2mm}
	      \includegraphics[width=1.0\textwidth]{images/HSCcosmos}\\
		      \column{0.5\textwidth}
		       \\
		       \vspace{-0.1cm}
		       HSC image (COSMOS) from g,r(1.5 hrs) ,i(3 hrs) PSF matched co-add ($\approx 27.5$)\\
		       Challenges:

\begin{itemize}
\item Unknown statistical distributions,   Truncated, censored and missing data, Unreliable quantities (e.g. unknown systematics and random errors)
\item PSF - short exposure - atmosphere dominated ?

\begin{itemize}
\item cosmic shear signal from weak lensing
\end{itemize}
\item Photometry  challenging - will Gaia help LSST ..
\item Everything is blended!!
\end{itemize}
		       \vspace{10mm}
		{\tiny       Processed with  {\em LSST Stack} \url{https://pipelines.lsst.io/}\\
	        \bf{Image HSC collaboration,  Robert Lupton}}
	       \end{columns}
       }




