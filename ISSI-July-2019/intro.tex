
\frame{\frametitle{ A little about myself}
	\vspace{-7pt}
\begin{itemize}
\item 1985ish started with BASIC on a  Commodore
\item 1993 MSc BSc Computer Science, University College Cork, Ireland
\item 1993 - 1997 Spacecraft Control Systems (C++), ESA ESOC  Germany
\item 1997 - 2001 Hipparcos, Integral, Planck, Gaia, Bepi-Sax  (C,Java,Oracle, HTM, HEALPix), ESA ESTEC Netherlands
\item 2001-2003 Commercial programming - some Astronomy (Java)
\item 2003-2005 The Johns Hopkins, SDSS and National Virtual Observatory (C,C\#,Java,Sqlserver)
\item 2005-2014 Gaia Astrometric Solution and Science Operations (Java, Oracle, Intersystems Cache)
\item 2012  PhD in Physics on Implementing the Gaia Astrometric Solution,  Barcelona University
\item 2014-2017 ESA Division head - all Science Ground Segments in Development
\item 2017- LSST Data Management Project Manager (Python,C++), Deputy Project Manager for Software (control systems)

\end{itemize}
}


\frame{\frametitle{ Gaia and  LSST continuum }
\begin{columns}
\column{0.4\textwidth}
\includegraphics[width=0.9\textwidth,trim=0 0 0 0] {images/GaiaLSSTaccuracy.png}\\
\column{0.6\textwidth}
\\
\vspace{-6.5cm}
\begin{itemize}
\item Gaia: excellent astrometry (and photometry), but only to $r >  20$
\item	LSST: photometry to $r < 27.5$ and time resolved measurements to $r <  24.5$
\item	Complementary: photometric, proper motion and trigonometric parallax errors are similar around r=20
\end{itemize}
\note{The Milky Way disk {\em belongs} to Gaia, and the halo to LSST (plus very faint and/or very red sources, such as white dwarfs and LT(Y) dwarfs).}
\end{columns}
}


\begin{agaframe}{Science topics - surveys}
\begin{tikzpicture}
    \node (im) {\includegraphics[height=1.0\textheight]{gaiaim/Science-return.png}};
    \node  at (im) {\includegraphics[height=2.20cm, angle=-12]{gaiaim/Gaia-model-2012.png}};
    \node (pl) [right=-0.5cm of im] {\includegraphics[width=9cm] {images/planckhistory}};
  \end{tikzpicture}
\end{agaframe}




