\section{Astrometry}
\frame {\frametitle{ Astrometric Solution}
Just one part of the Gaia processing !\\
From the Hipparcos catalogue
\cite[Volume 3 Chapter 23]{hip:catalogue}.

\begin{block}{Minimisation problem for astrometry}
\begin{equation}
	 ^{min}_{\bfvec{a},\bfvec{n}} \| \bfvec{g^{obs}} - \bfvec{g^{calc}}(\bfvec{a},\bfvec{n})\|_{M}
	\label{eq:generalobs}
\end{equation}
\end{block}
\begin{itemize}%[<+->]
\item $\bfvec{a}$  is the vector of unknowns describing a star's barycentric motion
represented by the measurement vector $\bfvec{g_{k}} = (G_{k},H_{k})^{\prime}$
and associated statistics.
\item $\bfvec{g^{obs}}$ represents the vector of all measurements
\item  $\bfvec{g^{calc}}$ represents the vector of detector
coordinates calculated from the astrometric parameters.
\item $\bfvec{n}$ is a vector
of nuisance parameters - required for realistic modeling (e.g. attitude, instrument calibration)

\item   $M$ metric defined by the statistics
of the data, (error weighting)

\end{itemize}
The complete new formulation for Gaia is in \citep{2012A&A...538A..78L}.
}


\frame {\frametitle{ Basic Gaia Problem }
%\vspace{-10pt}
Put more simply the data reduction must:\\
{\em
find the astrometric parameters (catalogue) best
predicting the ($10^{12}$) focal plane observations of the sources.
\citep{2011ExA....31..215O}
}
\begin{center}
\includegraphics[scale=0.6, clip=true,trim=1cm 0cm 0 1.5cm ]{gaiaim/predict}
\end{center}
}

